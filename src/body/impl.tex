\documentclass[../../main.tex]{subfiles}

\begin{document}

\ourchapter{Simplifying Parsers and Expressions}
Shit output from previous section. This motivates:
\begin{itemize}
  \item \cref{sec:simplify-parsers} discusses how parser terms can be simplified via domain-specific optimisations based on parser laws.
  \item \cref{sec:function-representation} discusses how expressions can be partially evaluated. This is achieved using another intermediate \textsc{ast}, this time based on the $\lambda$-calculus.
\end{itemize}

% TODO
% Writing domain-specific lint rules unlocks the potential for more powerful and interesting transformations utilising specialised domain knowledge.
% Desirable:
% * inspectability for analysis (that's what we're here for!) and optimisation
% The purpose of this chapter is to describe the intermediate representations of parsers (\cref{sec:parser-representation}) and functions (\cref{sec:function-representation}).
% Show that terms must be simplified to a normal form
% Demonstrate equivalence to dsl optimisations in staged metaprogramming
% Scalafix runs at the meta-level, outside of the phase distinction of compile- and run-time.
% Staged metaprogramming applies optimisations at compile-time, whereas these ``optimisations'' at applied post-compilation

\subfile{impl/parser}
\subfile{impl/expr}

\end{document}

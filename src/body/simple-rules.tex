\documentclass[../../main.tex]{subfiles}

\begin{document}

% TODO: Take inspiration from DLint paper, the Scala Refactoring master's thesis, HaRe papers
% Simple rules can consist of a single heading, containing:
% * Explanation of the rule
% * Simple example to show a diagnostic, and a before and after if it's fixable
% * How it's implemented in the code
% * Proof (if applicable)
% * Limitations

\ourchapter{Enforcing Idiomatic Usage of Scala Implicits}
Implicit conversions are a powerful feature in Scala, allowing users to supply an argument of one type when another is expected, to reduce boilerplate.
As noted by \textcite{willis_design_2022}, implicit conversions are particularly useful for designing \textsc{dsl}s.
In the context of parser combinators, they introduce the usage of implicit conversions to automatically lift string and character literals into parsers in the \emph{Implicit Conversions} design pattern.
This eliminates the need to explicitly wrap these elements in combinators:
\scala{string("parsley") | string("garnish")} can now be expressed as just \scala{"parsley" | "garnish"}, more closely resembling the style of a \textsc{bnf} grammar.

The \emph{Implicit Lexer} pattern is a further specialisation of this approach, hiding the boilerplate of whitespace handling entirely within a \scala{lexer} object.
This design pattern allows whitespace handling to be encapsulated as private combinators within the \scala{lexer} object, which are then made available only through implicit conversions automatically applied by the Scala compiler.

% TODO: tidy up and finish off this section intro

\section{Ambiguous Implicit Conversions}
\subsection*{Problem}
However, due to their utility, implicit conversions are also an easily abused feature of Scala.
They can obscure the flow of the program, making it difficult to understand what the code is doing and potentially hiding side effects or costly operations.
A downside particularly relevant to Parsley is that implicit conversions often lead to confusing error diagnostics when the compiler is unable to resolve them.

One common issue arises from ambiguous implicits when there are multiple implicit conversions in scope.
Parsley provides \scala{stringLift} and \scala{charLift} combinators in the \texttt{parsley.syntax.character} package for the \emph{Implicit Conversions} pattern,
and exposes an \scala{implicitSymbol} combinator for lexers to use in the \emph{Implicit Lexer} pattern.
For novice users, it is easy to accidentally import both sets of these implicits, when it is likely that they only intended to use the \scala{implicitSymbol} implicit. % TODO: explain why lexer implicit supercedes character implicits?
For example, consider the following code snippet:
\begin{minted}{scala}
val p = 'g' ~> "arnish"
p.parse("garnish")
// [error] type mismatch;
//   found   : String("arnish")
//   required: Parsley[?]
//  Note that implicit conversions are not applicable because they are ambiguous:
//   both method stringLift in object character of type (str: String): Parsley[String]
//   and method implicitSymbol in class ImplicitSymbol of type (s: String): Parsley[Unit]
//   are possible conversion functions from String("arnish") to Parsley[?]
//    val p = 'g' ~> "arnish"
//                   ^^^^^^^^
\end{minted}
%
Here, the compiler provides a detailed error message indicating the ambiguity between two possible implicit conversions.
However, the compiler is not always able to report such issues clearly. For instance, switching the position of the intended implicit conversion results in a less helpful message:
\begin{minted}{scala}
val p = "garnis" <~ 'h'
p.parse("garnish")
// [error] value <~ is not a member of String
//    val p = "garnis" <~ 'h'
//            ^^^^^^^^^^^
\end{minted}

\subsection*{Solution}
Ideally, this issue would be addressed by implementing a lint-on-compile rule, which could annotate the compiler error message at the exact location of the issue.
If this were implemented as a compiler plugin, partial information available from the compiler stages before the error could potentially provide enough detail to identify the exact clashing implicits.
This approach would allow leveraging domain knowledge to update the error message with more useful Parsley-specific diagnostics.

Incidentally, WartRemover has a related lint rule for implicit conversions\footnote{\url{http://www.wartremover.org/doc/warts.html#implicitconversion}},
although it only targets the locations where implicit conversions are \emph{defined}, not where they are \emph{applied}.
Despite this limitation, it serves as a proof of concept demonstrating the feasibility of such an approach.

Unfortunately, Scalafix restricts usage to only syntactic rules on the bare \textsc{ast} or semantic rules that operate fully post-compilation.
Since the ambiguous implicit conversions will cause compilation failures, this lint must be implemented as a syntactic rule.
Consequently, the solution takes a different approach: estimating the presence of clashing implicits by examining their import statements within each scope.

\subsection*{Example}
\Cref{fig:ambiguous-implicits-example} extends the previous example to a full Scala source file following the \emph{Implicit Lexer} pattern,
but where the user has erroneously additionally imported the \scala{stringLift} implicit from the \emph{Implicit Conversions} pattern.
This results in the Scala compiler throwing an error on line 6 due to ambiguous implicits.
When run on this file, \texttt{parsley-garnish} will report a warning similar to that shown in \cref{fig:ambiguous-implicits-warning}.

\begin{figure}[htbp]
\begin{subfigure}{\textwidth}
\begin{minted}[frame=single,linenos]{scala}
object parser {
  import parsley.syntax.character.stringLift
  import lexer.implicits._

  val p = "garnis" <~ 'h'
}

object lexer {
  import parsley.token.Lexer, parsley.token.descriptions.LexicalDesc

  private val lexer = new Lexer(LexicalDesc.plain)
  val implicits = lexer.lexeme.symbol.implicits
}
\end{minted}
\caption{A minimal Parsley program which fails to compile due to ambiguous implicits in the \texttt{parser} object.}
\label{fig:ambiguous-implicits-example}
\end{subfigure}
%
\begin{subfigure}{\textwidth}
\vspace{3ex} % TODO: ew
\begin{minted}[frame=single,fontsize=\small]{text}
warning: [AmbiguousImplicitConversions] This import may cause clashing implicit conversions:
* import parsley.syntax.character.stringLift at line 2
* import lexer.implicits._ at line 3
If this is the case, you may encounter confusing errors like 'method is not a member of String'.
To fix this, ensure that you only import a single implicit conversion.

  import lexer.implicits._
  ^^^^^^^^^^^^^^^^^^^^^^^^  
\end{minted}
\caption{The warning message reported at line 3 by the \emph{Ambiguous Implicit Conversions} lint rule.}
\label{fig:ambiguous-implicits-warning}
\end{subfigure}
\caption{Example of the \emph{Ambiguous Implicit Conversions} lint rule in action.}
\end{figure}

\subsection*{Implementation}
Unlike Java, Scala offers more flexibility with import statements, allowing them to appear anywhere in source files rather than just at the top.
Scala's import statements are lexically scoped, allowing their visibility to be limited to a single class, object, or function.
Additionally, Scala processes import statements in a top-down order within the file, further restricting their visibility, as scopes above an import cannot see the imports defined below them.

* Scalafix (via scalameta) provides a generic traversal of the AST: filter to find all import statements in top-down order
* This allows the scope to be lexically managed -- traversal in the same order that the compiler reads imports
* The ancestor AST node of an import statement is its enclosing scope
* Use ancestor information to determine which of the visited imports are in scope at that point

* to find stringLift: Pattern match to find if import is of form `import parsley.syntax.character.\_`
* to find implicit lexer: pattern match to find if there is an importee called `implicitSymbol` or if an import contains keywords `lexer` and `implicit(s)`

* if at any point in the traversal, both types of imports are in scope, report a warning

\section{Remove Explicit Usage of Implicit Conversions}

\end{document}

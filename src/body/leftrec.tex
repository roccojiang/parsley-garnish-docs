\documentclass[../../main.tex]{subfiles}

\begin{document}

\ourchapter{Removing Left-Recursion}\label{sec:factor-leftrec}

\section{Implementation}
\TODO{section intro}

\paragraph{Running example}
The following left-recursive parser and its transformation into its \scala{postfix} form will serve as a running example:
\begin{minted}{scala}
lazy val example: Parsley[String] = (example, string("a")).zipped(_ + _) | string("b")
\end{minted}

\subsection{The Need for an Intermediate \textsc{ast}}\label{sec:parser-ast-motivation}
The transformations described by \textcite{baars_leftrec_2004} require an explicit representation of the grammar and production rules so that they can be inspected and manipulated before generating code.
They achieve this by representing parsers as a deep-embedded datatype in the form of an intermediate \textsc{ast}, in a similar manner to \texttt{parsley}.

Since \texttt{parsley-garnish} is a linter, by nature, it has access to an explicit grammar representation in the form of the full Scala \textsc{ast} of the source program.
However, this \textsc{ast} is only a general-purpose representation of generic Scala code, rather than the \texttt{parsley} \textsc{dsl}.
Manipulating the \textsc{ast} becomes rather clumsy when trying to perform domain-specific operations on parsers.

Take for example the task of combining two \textsc{ast} nodes \scala{Term.Name("p")} and \scala{Term.Name("q")}, representing named parsers \scala{p} and \scala{q}, with the \emph{ap} combinator \scala{<*>}.
This operation can be concisely expressed with Scalameta quasiquotes, rather than manually writing out the full explicit \textsc{ast}:
\begin{minted}{scala}
q"p <*> q" ==
  Term.ApplyInfix(
    Term.Name("p"),
    Term.Name("<*>"),
    Type.ArgClause(Nil),
    Term.ArgClause(List(Term.Name("q")), None)
  )
\end{minted}
However, the  operation of inspecting the individual parsers \scala{p} and \scala{q} is not as straightforward.
Although quasiquotes can be used as extractor patterns in pattern matching, this usage is discouraged due to limitations in their design that makes it easy to accidentally introduce match errors\footnote{\url{https://scalameta.org/docs/trees/guide.html#with-quasiquotes-1}}.
Thus, extracting the parsers necessitates a long-winded pattern match like so:
\begin{minted}{scala}
val ap = SymbolMatcher.normalized("parsley.Parsley.`<*>`")

def deconstructAp(parser: Term) = parser match {
  case Term.ApplyInfix(p, ap(_), _, Term.ArgClause(List(q), _)) => (p, q)
}
\end{minted}
This involves dealing with abstract general-purpose syntax constructs like \scala{Term.ApplyInfix}, which are low-level details not relevant to the task of manipulating parsers.
Although this is not an issue for simple one-off transformations, for more specialised transformations like left-recursion factoring, it would be desirable to abstract away from these low-level syntactic details.
This motivates the need for an higher-level, intermediate \textsc{ast} representation that is more specialised to the domain of parser combinators.

\subsubsection{The Parser \textsc{adt}}
\texttt{parsley-garnish} therefore takes a similar approach as \textcite{baars_leftrec_2004} and \texttt{parsley} itself, building an intermediate \textsc{ast} as a deep-embedded parser combinator tree.
\Cref{fig:parser-adt} shows how this is implemented as a \scala{Parser} algebraic data type (\textsc{adt}).
All \scala{Parser} types represent \texttt{parsley} combinators, with the sole exception of \scala{NonTerminal} to represent references to named parsers.

\begin{figure}[htbp]
\begin{minted}{scala}
trait Parser
case class NonTerminal(ref: Symbol) extends Parser
case class Pure(x: Term) extends Parser
case object Empty extends Parser
case class Ap(p: Parser, q: Parser) extends Parser
case class Choice(p: Parser, q: Parser) extends Parser
\end{minted}
\caption{A subset of the core combinators in the \scala{Parser} \textsc{adt}.}
\label{fig:parser-adt}
\end{figure}

\paragraph{Deconstructing parsers}
Syntactic sugar for deconstructing parsers is provided by \scala{unapply} methods on symbolic extractor objects.
This makes pattern matching on parsers feel more natural:
\begin{minted}{scala}
object <*> {
  def unapply(parser: Ap): Option[(Parser, Parser)] = Some((parser.p, parser.q))
}

def deconstructAp(parser: Parser) = parser match {
  case p <*> q => (p, q)
}
\end{minted}

\paragraph{Constructing parsers}
Defining infix operators as extension methods on the \scala{Parser} trait provides syntactic sugar for constructing parsers:
\begin{minted}{scala}
implicit class ParserOps(private val p: Parser) extends AnyVal {
  def <*>(q: Parser): Parser = Ap(p, q)
  def <|>(q: Parser): Parser = Choice(p, q)
  def map(f: Term): Parser = FMap(p, f)
}

implicit class MultiParserOps(private val ps: List[Parser]) extends AnyVal {
  def zipped(f: Term): Parser = Zipped(f, ps)
}
\end{minted}
%
This makes working with \scala{Parser} terms feel closer to writing \texttt{parsley} code.
For example, notice how constructing the \emph{code} representation of the \scala{example} parser resembles how the original parser itself would be written:
\begin{minted}{scala}
val exNT = NonTerminal(Sym("path/to/package/ObjectName.example."))  

// val ex: Parsley[String] =     (ex,  string("a")).zipped(  _ + _ )  |  string("b")
   val ex: Parser          = List(exNT,   Str("a")).zipped(q"_ + _") <|>    Str("b")
\end{minted}

% Instead, represent parsers as an algebraic data type \textsc{adt} in the same way that Parsley itself uses a deep embedding to represent combinators as objects.
% Methods on these objects can then be used to manipulate them, and the resulting object can still be pattern matched, maintaining the static inspectability of the parsers.
% So then it's just like writing parsers in Parsley itself: \scala{p <*> q} constructs a \scala{Ap(p, q)} node which can still be pattern matched on.
% And similar to Parsley, representing everything as objects makes it easy to optimise using pattern matching on constructors.
% This representation also then gives us for free the implementation for lint rules such as \emph{Simplify Complex Parsers} rule, which applies parser laws to simplify parsers.

\subsection{Lifting to the Intermediate Parser \textsc{ast}}
Converting the raw Scala \textsc{ast} to the intermediate \textsc{ast} requires the following basic operations:
\begin{enumerate}
  \item Identifying all named parsers defined in the source program -- these correspond to non-terminal symbols in the grammar.
  \item Lifting the definition each parser into the intermediate \textsc{ast}, as a \scala{Parser} object.
  \item Collecting these into a map to represent the high-level grammar: the unique symbol of each named parser is mapped to its corresponding \scala{Parser} object, along with some extra meta-information required for the transformation.
\end{enumerate}
%
Most importantly, this meta-information includes a reference to a parser's original node in the Scala \textsc{ast}, so that any lint diagnostics or code rewrites can be applied to the correct location in the source file.
This is simply defined as:
\begin{minted}{scala}
case class ParserDefn(name: Term.Name, parser: Parser, tpe: Type.Name, originalTree: Term)
\end{minted}

\subsubsection{Identifying Named Parsers}
Finding \textsc{ast} nodes corresponding to the definition sites of named parsers involves pattern matching on \scala{val}, \scala{var}, and \scala{def} definitions with a type inferred to be some \scala{Parsley[_]}.
This type information is accessed by querying the Scalafix semantic \textsc{api} for the node's symbol information.
Consider the labelled \textsc{ast} structure of the \scala{example} parser:
\begin{minted}{scala}
// lazy val example: Parsley[String] = (example, string("a")).zipped(_ + _) | string("b")
// ^^^^     ^^^^^^^  ^^^^^^^^^^^^^^^   ^^^^^^^^^^^^^^^^^^^^^^^^^^^^^^^^^^^^^^^^^^^^^^^^^^
// mods      pats        decltpe                             rhs

val exampleTree = Defn.Val(
  mods = List(Mod.Lazy()),
  pats = List(Pat.Var(Term.Name("example"))),
  decltpe = Some(
    Type.Apply(Type.Name("Parsley"), Type.ArgClause(List(Type.Name("String"))))
  ),
  rhs = Term.ApplyInfix(...)
)
\end{minted}
%
% In this case, the type of \scala{example} is explicitly annotated by the user since this is required for recursive definitions.
% However in general, users will not explicitly annotate the types of their parsers, allowing the Scala compiler to infer the type.
Note that the \scala{decltpe} field refers to the syntax of the explicit type annotation, not the semantic information of the inferred type of the variable.
Therefore, this field will not always be present, so in the general case, the type must be queried via a symbol information lookup like so:
\begin{minted}{scala}
tree match {
  case Defn.Val(_, List(Pat.Var(varName)), _, body) =>
    println(s"qualified symbol = ${varName.symbol}")
    varName.symbol.info.get.signature match {
      case MethodSignature(_, _, returnType) =>
        println(s"type = $returnType")
        println(s"structure of type object = ${returnType.structure}")
    }
}
// qualified symbol = path/to/package/ObjectName.example.
// type = Parsley[String]
// structure of type object = TypeRef(
//   NoType,
//   Symbol("parsley/Parsley#"),
//   List(TypeRef(NoType, Symbol("scala/Predef.String#"), List()))
// )
\end{minted}
Seeing that the type of this \textsc{ast} node is \scala{Parsley[String]}, \texttt{parsley-garnish} can then proceed to convert the \scala{rhs} term into a \scala{Parser} \textsc{adt} object.
The map entry uses the fully qualified symbol for \scala{example} as the key, and the lifted \scala{Parser} object as the value.

% Thus, a full traversal through the source file builds a map of all named parsers, representing all non-terminals in the grammar defined within that file.

\subsubsection{Converting Scalameta Terms to the Parser \textsc{adt}}
Having identified the \textsc{ast} nodes which represent parsers, they need to be transformed into the appropriate \scala{Parser} representation.
This involves pattern matching on the \scala{scala.meta.Term} to determine which parser combinator it represents, and then constructing the appropriate \scala{Parser} instance.

Each \scala{Parser} defines a partial function \scala{fromTerm} to instantiate a parser from the appropriate \scala{scala.meta.Term}.
These \scala{fromTerm} methods perform the ugly work of pattern matching on the low-level syntactic constructs of the Scala \textsc{ast}.
All \scala{fromTerm} methods are combined to define the \scala{toParser} extension method on \scala{scala.meta.Term} -- this is where \textsc{ast} nodes are lifted to their corresponding \scala{Parser} representation.

The pattern matching example from \cref{sec:parser-ast-motivation} makes a reappearance in the definition of \scala{Ap.fromTerm}, where the arguments to the \scala{<*>} combinator are recursively lifted to \scala{Parser} objects:
% Use Scalafix's \scala{SymbolMatcher} to match tree nodes that resolve to a specific set of symbols.
% This makes use of semantic information from SemanticDB, so we are sure that a \scala{<*>} is actually within the \scala{parsley.Parsley} package, rather than some other function with the same name.
% This is much more robust compared to HLint, which suffers from false positives due to its reliance on syntactic information only.
\begin{minted}{scala}
// Type signatures in Parsley:
// p: Parsley[A => B], q: =>Parsley[A], p <*> q: Parsley[B]
case class Ap(p: Parser, q: Parser) extends Parser
object Ap {
  val matcher = SymbolMatcher.normalized("parsley.Parsley.`<*>`")

  def fromTerm: PartialFunction[Term, Ap] = {
    case Term.ApplyInfix(p, matcher(_), _, Term.ArgClause(List(q), _)) =>
      Ap(p.toParser, q.toParser)
  }
}
\end{minted}
%
Where a combinator takes a non-parser argument, this is treated as a black box and kept as a raw \textsc{ast} node:
\begin{minted}{scala}
// x: A, pure(x): Parsley[A]
case class Pure(x: Term) extends Parser
object Pure {
  val matcher = SymbolMatcher.normalized("parsley.ParsleyImpl.pure")

  def fromTerm: PartialFunction[Term, Pure] = {
    case Term.Apply(matcher(_), Term.ArgClause(List(expr), _)) => Pure(expr)
  }
}
\end{minted}

\subsubsection{Building the Grammar Map}
The overall process of converting the source file \textsc{ast} to a high-level map of the grammar can therefore be expressed as a single traversal over the \textsc{ast}:
\begin{minted}{scala}
object VariableDecl {
  def unapply(tree: Tree): ParserDefn = tree match {
    case Defn.Val(_, List(Pat.Var(varName)), _, body) if isParsleyType(varName) =>
      ParserDefn(
        name = varName,
        parser = body.toParser,
        tpe = getParsleyType(varName),
        originalTree = body
      )
    // similar cases for Defn.Var and Defn.Def
  }
}

val nonTerminals: Map[Symbol, ParserDefn] = doc.tree.collect {
  case VariableDecl(parserDef) => parserDefn.name.symbol -> parserDef
}.toMap
\end{minted}

\subsection{Implementing the Left-Recursion Transformation}
\TODO{TODO}


\end{document}
